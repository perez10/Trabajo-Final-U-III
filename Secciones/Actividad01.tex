
\documentclass[conference]{IEEEtran}
\usepackage{blindtext, graphicx}
\usepackage[utf8]{inputenc}
\usepackage[spanish, activeacute]{babel} 
\usepackage{listings}
\usepackage{amssymb}
\usepackage[breaklinks=true]{hyperref}
\pagestyle{myheadings}
\markright{Estrategias de seguridad en base de datos}
\hyphenation{op-tical net-works semi-conduc-tor}

\begin{document}

\title{ESTRATEGIAS DE SEGURIDAD EN BASE DE DATOS}
% make the title area
\maketitle


\begin{abstract}
Es indudable que cada dia las entidades dependen de mayor medida de la información y de la tecnologia, y que los sistemas de información están más soportadas por la tecnologia,frente a la realidad de hace pocas décadas.
\\
Por otra parte, hace unos años la protección era más facil,con arquitecturas centralizadas y terminales no inteligentes , pero hoy en día los entornos son realmente complejos, con diverdidad de plataformas y proliferación de redes, no sólo internas sino también externas, incluso con enlaces internacionales.
\\
Entre las plataformas físicas (hardware) pueden estar: ordenadores grandes y medios ordenadores departamentales y personales, solos o formando parte de red, e incluso ordenadores portátiles. Esta diversidad acerca la información a los usuarios, si bien hace mucho más dificil proteger los datos, especialmente porque los equipos tienen filosofias y sistemas operativos diferentes, incluso a veces siendo del mismo fabricante.
\\
Al hablar de seguridad hemos preferido centrarnos en la información misma, aunque a menudo se hable de seguridad informatica, de seguridad de los sistemas de información o de seguridad de las tecnologias de la información.
\\
En cualquier caso hay tres aspectos principales, como distintas vertientes de la seguridad.
\\
\end{abstract}

\begin{abstract}

Undoubtedly, every day the entities depend more on information and technology, and that information systems are more supported by technology, compared to the reality of a few decades ago.
\\
On the other hand, a few years ago the protection was easier, with centralized architectures and non-intelligent terminals, but nowadays the environments are really complex, with divergence of platforms and proliferation of networks, not only internal but also external, even with links international
\\
Among the physical platforms (hardware) can be: large computers and departmental and personal computers, alone or as part of the network, and even laptops. This diversity brings information closer to users, although it makes it much more difficult to protect data, especially because computers have different philosophies and operating systems, sometimes even from the same manufacturer.
\\
When talking about security, we have preferred to focus on the information itself, although there is often talk about information security, security of information systems or security of information technologies.
\\
In any case there are three main aspects, as different aspects of security.
\end{abstract}

% Note that keywords are not normally used for peerreview papers.
\begin{IEEEkeywords}
seguridad base de datos, MySQL, MongoDB.
\end{IEEEkeywords}


\IEEEpeerreviewmaketitle


\section{Introducci\'on}
La gran mayoría de los datos sensibles del mundo están almacenados en sistemas gestores de bases de datos comerciales tales como Oracle, Microsoft SQL Server entre otros, y atacar una bases de datos es uno de los objetivos favoritos para los criminales.
\\
Esto puede explicar por qué los ataques externos, tales como inyección de SQL, subieron 345 por ciento en 2009, “Esta tendencia es prueba adicional de que los agresores tienen éxito en hospedar páginas Web maliciosas, y de que las vulnerabilidades y explotación en relación a los navegadores Web están conformando un beneficio importante para ellos”[*]
\\
Para empeorar las cosas, según un estudio publicado en febrero de 2009 The Independent Oracle Users Group (IOUG), casi la mitad de todos los usuarios de Oracle tienen al menos dos parches sin aplicar en sus manejadores de bases de datos [1].
\\
 Mientras que la atención generalmente se ha centrado en asegurar los perímetros de las redes por medio de, firewalls, IDS / IPS y antivirus, cada vez más las organizaciones se están enfocando en la seguridad de las bases de datos con datos críticos, protegiéndolos de intrusiones y cambios no autorizados
\\
En las siguientes secciones daremos las siete recomendaciones para proteger una base de datos en instalaciones tradicionales..
\\

\section{CONCEPTO}

La seguridad en las base de datos es un mecanismo fundamental ya que todo de sistema informatizado esta expuesto a cualquier tipo de amenazas de daño, enormes y desastrosas como pequeñas y leves pero que de una manera u otra causan perdida de confidencialidad
\\
\begin{flushright}
  \includegraphics[scale=0.31]{Imagenes/seguridadbd.jpg}
\end{flushright}
\centering
\\Figura 1
\\

\begin{itemize}
\item \textbf{AMENAZAS:} Se deben considerar las amenazas para cada tipo de empresa donde se implementara el sistema de base de datos, ya que pueden haber amenazas particulares a las que se este mas expuesto.
\end{itemize}

\section{PRINCIPIOS BÁSICOS PARA LA SEGURIDAD}

Suponer que el diseño del sistema es público:
\begin{itemize}
\item \textbf{} El defecto debe ser: sin acceso.
\item \textbf{} Chequear permanentemente.
\item \textbf{} Los mecanismos de protección deben ser simples, uniformes y construidos en las capas más básicas del sistema.
\end{itemize}

\section{MEDIDAS DE SEGURIDAD}

Suponer que el diseño del sistema es público:
\begin{itemize}
\item \textbf{FÍSICAS:} Controlar el acceso al equipo.
\\
mediante tarjetas de acceso.
\item \textbf{PERSONAL:} Acceso solo de personal autorizado.
\\
identificación directa de personal.
\item \textbf{SGBD:} Uso de herramientas que proporcione el SGBD.
\\
perfiles de usuario, vistas, restricciones de uso de vistas.
\\
\end{itemize}
\\

\section{REQUISITOS PARA LA SEGURIDAD DE LAS BD}
\begin{itemize}
\item \textbf{} La base de datos debe ser protegida contra el fuego, el robo y otras formas de destrucción.
\\
\item \textbf{} Los datos deben ser reconstruibles, ya que siempre pueden ocurrir accidentes.
\\
\item \textbf{} Los datos deben poder ser sometidos a procesos de auditoria.
\\
\item \textbf{} El sistema debe diseñarse a prueba de intromisiones, no deben poder pasar por alto los controles.
\\
\item \textbf{} Ningún sistema puede evitar las intromisiones malintencionadas, pero es posible hacer que resulte muy difícil eludir los controles.
\\
\item \textbf{} El sistema debe tener capacidad para verificar que sus acciones han sido autorizadas
\\
\item \textbf{} Las acciones de los usuarios deben ser supervisadas, de modo tal que pueda descubrirse cualquier acción indebida o errónea.
\\
\end{itemize}

\section{CARACTERÍSTICAS PRINCIPALES}
El objetivo es proteger la Base de Datos contra
accesos no autorizados.
Las 3 principales caracteristicas de la seguridad en la base de datos son:
\begin{itemize}
\item \textbf{}La Confidencialidad de la información
\item \textbf{}La Integridad de la información
\item \textbf{}La Disponibilidad de la información
\end{itemize}
\\


\section{TECNOLOGIAS USADAS}
\subsection{MongoDB}
MongoDB es una base de datos documental y \'agil de c\'odigo abierto que permite a los esquemas cambiar r\'apidamente cuando las aplicaciones evolucionan, proporcionando siempre la funcionalidad que los desarrolladores esperan de las bases de datos tradicionales, tales como \'indices secundarios, un lenguaje completo de b\'usquedas, consistencia estricta y un alto rendimiento.\cite{mongo}
\\
\begin{flushright}
\includegraphics[scale=0.15]{Imagenes/mongodbsql.jpg}
\end{flushright}
\centering
\\Figura 2
\subsection{Neo4j}
Neo4j es una base de datos gr\'afica nativa altamente escalable, diseñada espec\'ificamente para aprovechar no s\'olo los datos, sino tambi\'en sus relaciones.

El motor de procesamiento y procesamiento de grafos nativo de Neo4j proporciona un rendimiento constante y en tiempo real, ayudando a las empresas a crear aplicaciones inteligentes para satisfacer los actuales desafíos de datos en constante evolución.\cite{neo}
\\\\
\begin{flushright}
\includegraphics[scale=2.5]{Imagenes/neo4j.png}
\end{flushright}
\centering
\\Figura3
\subsection{Neo4j Doc Manager}
Es una herramienta que permite migrar documentos desde MongoDB a una estructura de grafos de propiedades de Neo4j. Simplemente se ejecuta en segundo plano y la informaci\'on que se encuentra en MongoDB se importa a un grafo visible desde Neo4j, para permitir a los desarrolladores de MongoDB almacenar datos JSON en Mongo mientras consulta las relaciones entre los datos usando Neo4j.

MongoDB almacena datos como documentos similares a JSON, mientras que Neo4j almacena datos como grafos de propiedades. Para permitir la consulta basada en grafos de datos MongoDB, necesitamos determinar c\'omo mapear entre estas dos estructuras de datos diferentes. Neo4j Doc Manager provee un plan de mapeo por defecto. Siguiendo la convenci\'on en lugar de requerir configuraci\'on.\cite{doc}
\\
\begin{flushright}
\includegraphics[scale=0.25]{Imagenes/mongo_connector.png}\\
\end{flushright}
\centering
Figura 4
\section{Caso de Estudio: MongoDB y Neo4j}
Al crear aplicaciones lo desarrolladores tienen una gran variedad de tecnolog\'ias para escoger, incluyendo a lo que la elecci\'on de base de datos  concierne, sin embargo, al a\~nadir nuevas tecnolog\'as a menudo termina significando mayor complejidad en el manejo de los sistemas en lugar de obtener un beneficio significante. La idea de la persistencia pol\'iglota es permitir tomar ventaja de los puntos fuertes de las distintas capas de persistencia para mejorar la funcionalidad de las aplicaciones. A continuaci\'on se presenta un caso de estudio tomado de la p\'agina oficial de Neo4j \cite{caso} basado en una aplicaci\'on de comercio electr\'onico en el que se utiliza MongoDB y Neo4j juntos en base a los puntos fuertes de cada base de datos y tambi\'en se utiliza Neo4j Doc Manager para Mongo Connector, que permite la sincronizaci\'on en tiempo real de MongoDB a Neo4j.

\begin{figure}[!h]
\centering
\includegraphics[width=0.4\textwidth]{5}
\caption{Se utiliza una BD claves-valor para alimentar el carrito de compras del usuario, una BD orientada a  documentos para la b\'usqueda y la navegaci\'on por el cat\'alogo de productos y una BD orientada a grafos para obtener recomendaciones personalizadas en tiempo real.}
\label{fig5}
\end{figure}

Un caso de uso b\'asico en aplicaciones de comercio electr\'onico para una base de datos documental es la de manejar un cat\'alogo de productos y poder realizar b\'usquedas sobre \'el. Entre los requisitos funcionales está el soportar una diversa cartera de productos con consultas complejas y filtrado a trav\'es de los atributos de muchos productos. Para entender c\'omo el uso de una base de datos de grafos  junto con una base de datos de documentos puede mejorar una aplicaci\'on, se tomar\'a como ejemplo una aplicaci\'on web que ofrece un cat\'alogo de cursos en l\'inea.

\begin{figure}[!h]
\centering
\includegraphics[width=0.5\textwidth]{6}
\caption{Un cat\'alogo de cursos en l\'inea es un buen caso de uso de una base de datos de documentos.}
\label{fig6}
\end{figure}

La vista mostrada permite al usuario ver una lista de todos los cursos disponibles para ellos, buscar por palabra clave y filtrar por fecha, idioma o categor\'ia. La vista tiene que ser alimentada de una s\'ola consulta en BD por lo que se est\'a volcando una gran cantidad de informaci\'on ah\'i. Las consultas tienen que utilizar diferentes tipos de \'indices (de texto completo, filtro por categor\'ias, filtros por intervalo de tiempo) y devolver toda la informaci\'on necesaria para renderizar la vista, sin embargo, falta mostrar informaci\'on que est\'e acorde al contexto del usuario. La aplicaci\'on debe saber que cursos el usuario ha tomado anteriormente y c\'omo el usuario ha interactuado con otros usuarios en la plataforma para ser capaces de ofrecer alg\'un contenido personalizado basado en estas preferencias del usuario, por lo que entra en juego lo que ser\'ia los cursos recomendados al usuario en cuesti\'on.

\begin{figure}[!h]
\centering
\includegraphics[width=0.5\textwidth]{7}
\caption{Cat\'alogo de cursos con contenido personalizado, tales como las recomendaciones de cursos en base a la informaci\'on que se sabe acerca de las preferencias del usuario actual.}
\label{fig7}
\end{figure}

Por sus caracter\'isticas Neo4j es muy bueno para la generaci\'on de recomendaciones en tiempo real. MongoDB podr\'ia ser la m\'as id\'onea en servir a un cat\'alogo de productos, pero si lo que se quiere es generar contenido centrado en el usuario como por ejemplo las recomendaciones personalizadas de productos se sabe que es f\'acil en Neo4j. Ahora le pregunta es ¿C\'omo?, normalmente, esto implicar\'ia la sincronizaci\'on de datos en la capa de aplicaci\'on, es decir, interactuar directamente con  MongoDB, e interactuar directamente con Neo4j pero surgen otras interrogantes como: ¿Ambas operaciones terminaron con \'exito? ¿Qu\'e hacer si una de las transacciones falla? ¿En qu\'e punto se sincronizan los datos? Esto puede convertirse r\'apidamente en un componente muy complicado. Los beneficios de la persistencia pol\'iglota se producen a expensas de la complejidad. Ahora tenemos que aplicar la l\'ogica a nivel de aplicaci\'on para escribir datos tanto MongoDB y Neo4j.

\subsection*{Neo4j Doc Manager: Habilitaci\'on de Persistencia Pol\'iglota en MongoDB y Neo4j}

El proyecto Neo4j Doc Manager de los creadores de Neo4j es una implementaci\'on del proyecto Mongo Connector, proporcionado por la gente de MongoDB. Mongo Connector proporciona un mecanismo para notificar las solicitudes y actualizaciones en MongoDB y escribir los cambios a un sistema de destino. Neo4j Doc Manager funciona mediante el uso del Oplog (un registro de todas las operaciones en MongoDB). Siempre que hay una actualizaci\'on de un documento (por ejemplo, una inserci\'on, actualizaci\'on o eliminaci\'on), Neo4j Doc Manager es notificado de la actualizaci\'on y \'este contiene la l\'ogica para convertir ese documento en un modelo de grafos con sus respectivas propiedades y, a continuaci\'on, escribe inmediatamente esa actualizaci\'on para Neo4j.

\begin{figure}[!h]
\centering
\includegraphics[width=0.5\textwidth]{8}
\caption{Neo4j Doc Manager es notificado de todas las operaciones en MongoDB, convierte esos cambios a un modelo de grafos  y de inmediato escribe esos cambios a Neo4j.}
\label{fig8}
\end{figure}

El Neo4j Doc Manager permite al desarrollador de aplicaciones alcanzar el objetivo de construir una aplicaci\'on que aprovecha la persistencia pol\'iglota con MongoDB y Neo4j. En el contexto del cat\'alogo de cursos, perfectamente se puede proporcionar b\'usqueda y navegaci\'on entre los productos con MongoDB, y proporcionar recomendaciones en tiempo real en base a lo que los cursos que el usuario ha tomado y c\'omo han interactuado con el sistema desde Neo4j.

\subsection*{Neo4j Doc Manager: Instalaci\'on e Inicio del servicio}

Ya el equipo en d\'onde se ejecutar\'a el servicio debe tener instalado y configurado MongoDB y Neo4j.\\
Se instala Neo4j Doc Manager clonando el repositorio y estableciendo el PYTHONPATH al directorio local correspondiente:

\begin{lstlisting}
git clone https://github.com/neo4j-contrib/
          neo4j_doc_manager.git
cd neo4j_doc_manager
export PYTHONPATH=.
\end{lstlisting}

Si se tiene la autenticaci\'on habilidada para Neo4j, se debe asegurar que la variable de ambiente  correspondiente contenga el nombre de usuario y la contraseña para conectarse al servidor:

\begin{lstlisting}
export NEO4J_AUTH=<user>:<password>
\end{lstlisting}

Si la autenticaci\'on est\'a deshabilitada en Neo4j esa acci\'on no es requerida. Para deshabilitar la autenticación, se tiene que ir hasta el directorio de instalaci\'on de Neo4j y editar el archivo ' conf/neo4j-server.properties'  de la siguiente manera:

\begin{lstlisting}
dbms.security.auth_enabled=false
\end{lstlisting}

Ahora, de forma adicional se debe asegurar que mongo est\'e corriendo en un conjunto de r\'eplicas. Para iniciar y establecer una réplica en Mongo basta con:

\begin{lstlisting}
mongod --replSet myDevReplSet
\end{lstlisting}

Luego, ya con el servicio iniciado, abrir una c\'onsola de Mongo y correr:

\begin{lstlisting}
rs.initiate()
\end{lstlisting}

Por \'ultimo, una vez realizado los pasos anteriores se inicia el servicio de Neo4j Doc Manager con el siguiente comando en una s\'ola l\'inea:

\begin{lstlisting}
mongo-connector -m localhost:27017 -t
http://localhost:7474/db/data -d
neo4j_doc_manager
\end{lstlisting}

-m espec\'ifica el servidor de MongoDb, -t el servidor de Neo4j (incluyendo el protocolo) y -d espec\'ifica el Neo4j Doc Manager.

\subsection*{Neo4j Doc Manager: Primeros pasos}

Los documentos se convierten en grafos en base a la estructura del documento. Las llaves del documento se convertir\'an en nodos. Los valores anidados entre cada llave se convertir\'an en propiedades.

A continuaci\'on se muestra un documento que se inserta en MongoDB para ejemplificar como trabaja el Neo4j Doc Manager.

\begin{figure}[!h]
\centering
\includegraphics[width=0.5\textwidth]{10}
\caption{Documento insertado en MongoDB que se convierte a grafos en Neo4j en base a su estrcutura}
\label{}
\end{figure}

El Documento insertado en MongoDB es transformado a un modelo de grafos en Neo4j.\\

\begin{figure}[!h]
\centering
\includegraphics[width=0.3\textwidth]{11}
\caption{Informaci\'on de la base de datos en Neo4j}
\label{}
\end{figure}

\begin{figure}[!h]
\centering
\includegraphics[width=0.45\textwidth]{12}
\caption{Grafo en Neo4j que se gener\'o seg\'un el documento insertado en MongoDB}
\label{}
\end{figure}

Nodos creados:\\
\begin{itemize}
\item producto: producto es el nodo ra\'iz, viene del nombre de la colecci\'on a la que pertenece el documento con un  ''\_id'' que tambi\'en viene de MongoDB. Documentos no anidados son convertidos a propiedades del nodo, como por ejemplo: ''nombre'',''descripcion'', ''caracteristicas'',''sku'.
\item envio: envio es un documnto embebido. Pares claves-valor que est\'en dentro son convertidos a propiedades del nodo como por ejemplo: ''peso'', ''ancho'', ''alto'', ''profundidad''. Tambi\'en tiene una propiedad ''\_id'' que viene del nodo ra\'iz y es la mismo.
\item precio: precio es un documento embebido por lo que es análogo al nodo session.
\item categoria: en la colecci\'on producto hay una clave ''categoria\_id'' que Neo4j Doc Manager toma como una referencia a un nodo con etiqueta categoria que contenga ese id y crea la relaci\'on entre ellos. En caso de no existir, crea el nodo categoria y le asigna el id que tiene como valor en el documento.
\end{itemize}
También es importante mencionar que a los nodos se les agrega autom\'aticamente una propieda ''\_ts'' que representa su timestamp de la creaci\'on en MongoDB.\\

Relaciones creadas:\\

\begin{itemize}
\item producto\_precio: Una relaci\'on que conecta a los nodos producto y precio.
\item producto\_envio: Una relaci\'on que conecta a los nodos producto y envio.
\item producto\_categoria: Una relaci\'on que conecta a los nodos producto y categoria.\\
\end{itemize}

Ya visto como se hace la conversi\'on de un documento en MongoDB a grafos en Neo4j podr\'ia decirse que las posibilidades son muchas, ya que desde MongoDB se puede trabajar directamente con las colecciones y los datos se ver\'an reflejados en Neo4j para despu\'es procesar consultas relacionas a estructuras de grafos. Desde Mongo se puede hacer actualizaciones y eliminaciones  seg\'un criterios de filtrado que se reflejar\'an en Neo4j y que permitir\'an hacer cosas como por ejemplo: agregar o eliminar propiedades en los nodos, cambiar la estructura de los mismos, agregar o eliminar relaciones, entre otras cosas. Para ello, se invita al lector a profundizar m\'as en el tutorial de Neo4j Doc Manager que se encuentra en la p\'agina oficial de Neo4j  y en donde se explica de manera detallada los comandos a ingresar en MongoDb y los resultados obtenidos. Tutorial disponible en el siguiente enlace:\\
\url{https://neo4j.com/developer/neo4j-doc-manager/}

\subsection*{Neo4j Doc Manager: Aplicaci\'on de Persistencia Pol\'iglota}

Siguiendo el esquema propuesto para una aplicaci\'on de persistencia pol\'iglota  los autores del presente documento desarrollar\'on una sencilla aplicaci\'on de comercio electr\'onico que maneja datos provenientes de diferentes sistemas manejadores de base de datos para su funcionamiento.

\begin{figure}[!h]
\centering
\includegraphics[width=0.5\textwidth]{13}
\caption{Vista general de la aplicaci\'on de comercio electr\'onico}
\label{}
\end{figure}

Las secciones de categor\'ias y de cat\'alogo de productos se alimentan de datos almacenados en MongoDB. Se muestra contenido adecuado al contexto del usuario mediante recomendaciones que se hacen a trav\'es de consultas hechas en Neo4j y se cuenta con un carrito de compras (Figura 16) que se puede implementar mediante una base de datos clave-valor.

\begin{figure}[!h]
\centering
\includegraphics[width=0.5\textwidth]{14}
\caption{Cat\'alogo de productos de la aplicaci\'on de comercio electr\'onico con datos obtenidos de MongoDB}
\label{}
\end{figure}

\begin{figure}[!h]
\centering
\includegraphics[width=0.5\textwidth]{15}
\caption{Productos recomendados al usuario en la aplicaci\'on de comercio electr\'onico con datos obtenidos de Neo4j}
\label{}
\end{figure}

\begin{figure}[!h]
\centering
\includegraphics[width=0.45\textwidth]{16}
\caption{Carrito de compras de la aplicaci\'on de comercio electr\'onico}
\label{}
\end{figure}


\section{Conclusiones}
  La base de datos a hecho avances significativos en el manejo de la seguridad de las bases de datos.
  \\
  Varias de las aplicaciones que manejamos en nuestro diario vivir que requieren confidencialidad necesitan modelos mas sofisticados de seguridad: Entidades bancarias, medicas, departamentos gubernamentales, inteligencia militar, etc.
  \\
  Aunque la implementación de seguridad mas sofisticada no es tarea fácil, debemos hacer el esfuerzo por lograr que nuestros datos estén completamente seguros


\ifCLASSOPTIONcaptionsoff
  \newpage
\fi


\begin{thebibliography}{1}

\bibitem{SeguridadTecnologica}
ISO/IEC 27001:2005 - Information technology -- Security    techniques [en]
\\
http://www.iso.org/iso/catalogue-detail?Csnumber=42103.

\bibitem{sistemaAlmacenamiento}
Pramod J. Sadalage Martin Fowler. NoSQL Distilled: A brief Guide to the Emerging
World of Polyglot Persistence. Addison-Wesley, 2013.

\bibitem{BigTable}
Fay Chang, Jeffrey Dean, Sanjay Ghemawat, Wilson C. Hsieh, Deborah A. Wallach, Mike Burrows, Tushar Chandra, Andrew Fikes, and Robert E. Gruber. 2008. Bigtable: A Distributed Storage System for Structured Data. ACM Trans. Comput. Syst. 26, 2, Article 4 (June 2008), 26 pages. DOI=http://dx.doi.org/10.1145/1365815.1365816

\bibitem{NoSQL}
Vaish, Gaurav. Getting Started with NoSQL. Packt Publishing, 2013.

\bibitem{Json}
ECMA-404. Introducción a JSON. 2013. [online]
Disponible en: http://www.json.org/json-es.html

\bibitem{wanderu}
Boyd R. Polyglot Persistence Case Study: Wanderu + Neo4j + MongoDB Neo4j Blog, 2015.

\bibitem{zephyr}
Chaudhari M. Integrating Diverse Healthcare Data using MongoDB and Neo4j.Neo4j Blog, 2016.

\bibitem{mongo}
MongoDB, Inc. Reinventando la gestión de datos. [online]
Disponible en: https://www.mongodb.com/es

\bibitem{neo}
Neo Technology, Inc. Neo4j: The World’s Leading Graph Database. [online]
Disponible en: https://neo4j.com/product/

\bibitem{doc}
Neo Technology, Inc. Neo4j: The World’s Leading Graph Database. [online]
Disponible en: https://neo4j.com/developer/neo4j-doc-manager/

\bibitem{caso}
Lyion W. Neo4j Doc Manager: Polyglot Persistence for MongoDB and Neo4j. Neo4j Blog, 2015.

\end{thebibliography}

\begin{IEEEbiography}[{\includegraphics[width=1in,height=1.25in,clip,keepaspectratio]{picture}}]{John Doe}
\blindtext
\end{IEEEbiography}



\end{document}
